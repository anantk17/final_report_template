\section{Related Work}

\subsection{System Auditing}
%Why auditing is relevant in today's world.
Due to its value in threat detection and investigation, system auditing is the subject of renewed interest in traditional systems. While a number of experimental audit frameworks have incorporated notions of data provenance \cite{Bates2015},\cite{Pasquier2017b},\cite{Pohly2012},\cite{Pasquier2017a}, the bulk of this work is also based on commodity audit frameworks such as Linux Audit. Techniques have also been proposed to extract threat intelligence from large volumes of log data \cite{Pei2016,203676,Kwon2018,Hassan2019}. In this work, we seek to understand if such techniques are directly applicable to RTS while adhering to the unique temporal and resource constraints of the domain. Our study finds that existing auditing solutions may not be directly suited for RTS and there is a need for solutions that reduce the amount of audit logs generated by building on the notion of execution partitioning of log activity \cite{Kwon2018,Hassan2019,Hassan2020,Ma2017,Lee2013a} to remove redundant data.

\subsection{Forensic Reduction}
 A lot of work has gone into improving the cost-utility ratio of system auditing by pruning, summarizing, or compressing audit data that is unlikely to be used during investigations \cite{Tang2018,Xu2016,Ma2015,Ben2018,Hassan2018a,Bates2015a,Bates2017b,Hossain2018,Lee2013}. Of these, Ma et al.'s KCAL~\cite{Ma2018} and ProTracer~\cite{Ma2016} systems are a few that inline their reduction methods into the kernel. These approaches only consider the impact of synchronous auditing overheads on system performance, but in order to design a long-running RTS, we need to understand system behavior over long runs which includes the log maintenance overhead. Our findings suggest that userspace reduction solutions might not be effective in RTS as resource constraints might cause loss of audit information in the kernel. Although it is unclear if the existing kernel based approaches are suitable for real-time applications, inline kernel log reduction seems to provide a promising path forward to enable efficient system auditing in RTS.

\subsection{Auditing RTS} Although auditing has been widely acknowledged as an important aspect of securing embedded devices \cite{embedded_audit_1,embedded_audit_2,embedded_audit_3},
the unique challenges that arise when auditing RTS have received only limited attention.
Wang et al. present ProvThings, an auditing framework for monitoring IoT smart home deployments \cite{Wang2017},
  but rather than audit low-level embedded device activity their system monitors API-layer flows on the IoT platform's cloud backend.
Tian et al. present a block-layer auditing framework for portable USB storage  that can be used to diagnose integrity violations \cite{Tian2016}.
Their embedded device emulates a USB flash drive, 
  but does not consider system call auditing of real-time applications.
Wu et al. present a network-layer auditing platform that captures the temporal properties of network flows and can
  thus detect temporal interference \cite{Wu2019}.  
Whereas their system uses auditing to diagnose performance problems in networks,
  the present study considers the performance problems created by auditing within real-time applications.
In contrast to these systems, this study highlights
  the challenges of RTS auditing by attempting to naively incorporate the vanilla Linux Audit system into the real-time task schedule.